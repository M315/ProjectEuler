
%=============================================================================
%
%	Apuntes de la asignatura introductoria del Master de Big Data EAE
%
%=============================================================================



%=============================================================================
%	Packages
%=============================================================================

\documentclass[10pt, a4paper]{article}		% document format
\usepackage{graphicx}						% package to insert graphics
\usepackage[english]{babel}
\usepackage[utf8]{inputenc}					% this enables all unicode characters
\usepackage{amsmath, amsfonts, amssymb} 	% maths packages
\usepackage{amsthm}							% theorem styles
\usepackage{fancyhdr}						% package for headers, foot and matgin notes
\usepackage{enumitem}						% allows to change the enumeration format in line
\usepackage{mathtools}                      % useful math symbols like floor



%=============================================================================
%	Commands
%=============================================================================

\providecommand{\U}[1]{\protect\rule{.1in}{.1in}}
\newcommand{\N}{\mathbb{N}}
\newcommand{\E}{\mathcal{E}}



%=============================================================================
%	Environment
%=============================================================================

\theoremstyle{plain}
\newtheorem{exercise}{Exercise}
\newtheorem{algorithm*}{Algorithm}

\theoremstyle{definition}
\newtheorem{solution}{Solution}[exercise]
\renewcommand*{\thesolution}{\theexercise.\alph{solution}}	% Make solution instances enumerate with number of exercise and letter of solution




%============================================================================
%	Pagestyle
%=============================================================================

\pagestyle{fancy}		% As we use the fancyhdr pkg need the fancy pagestyle, for more information:
						% http://tug.ctan.org/tex-archive/macros/latex/contrib/fancyhdr/fancyhdr.pdf


%\topmargin 0 pt
%\oddsidemargin 0 pt
%\evensidemargin 0 pt
%\marginparwidth 0 pt


%\textheight 25cm
%\textwidth 400 pt
%\advance\textheight by \topskip

%\parindent 0pt
%\parskip 5mm plus 1mm minus 1mm

%\fancyhead[C]{\textbf{el que sigui}}					% Central header
\fancyhead[L]{Project Euler 86}			% Left header
\fancyhead[R]{Marc Jovani Bertran}						% Right header



%=============================================================================
%	Document
%=============================================================================

\begin{document}

\section{Find the lenght}

We want to know how to compute the lenght of the shortest stright line in a cuboid given $a \geq b \geq c \geq 1$.
To do it, we select a point $x$ in the some edge, let us start with $a$ so we get $x : a - x$.

\begin{equation}
    L(x) = \sqrt{b^2 + x^2} + \sqrt{c^2 + (a - x)^2}
\end{equation}

Let's find the minimum:

\begin{equation}
    L'(x) = \frac{a - x}{\sqrt{c^2 + (a - x)^2}} - \frac{x}{\sqrt{b^2 + x^2}} = 0
    \quad\Longleftrightarrow\quad
    x = \frac{ab}{b + c}
\end{equation}

Now we have that the lenght of the shortest path is

\begin{equation}
    L
    = \sqrt{b^2 + \frac{a^2 b^2}{(b + c)^2}} + \sqrt{c^2 + \frac{a^2 c^2}{(b + c)^2}}
    = \sqrt{a^2 + b^2 + c^2 + 2bc}
\end{equation}

Note that as $a \geq b \geq c \geq 1$ this length is the shortest cause if we compute the length selecting $x$ in $b$ we will end up with $\sqrt{a^2 + b^2 + c^2 + 2ac}$ and similarly with $c$.
So as $bc \leq ac \leq ab$ we have that $L$ is the distance of the shortest path in the cuboid.

\section{Pythagorean triplets}

Note that $L$ can be writen as a pythagorean triplet

\begin{equation}
    L
    = \sqrt{a^2 + (b + c)^2}
\end{equation}

So the approach to find a solution is to compute all the pitagorean triplets $(x,y,z)$ were $x$ and $y$ are smaller than $M$.

To generate all the primitive pythagorean triplets we use Euclid's formula. Given $m > n > 0$, then

\begin{equation}
    x = m^2 - n^2, \quad
    y = 2mn, \quad
    z = m^2 + n^2,
\end{equation}

form a pythagorean triplet. And if $m$ and $n$ are coprimes not both odd then the triplet is primitive.

\section{Code}

First we create a function $pythagorean\_triplets$ that generates all the pythagorean triplets, using fast gcd, with $z$ smaller than some given constant.

Then we create a funcition that given $a$ and $x = b + c$ computes all the possible cuboids with that $L$, that are smaller than $M$.

To do so we check that $a \leq m$ then if $x \leq a$ we have $\lfloor\frac{x}{2}\rfloor$ solutions as we are looking the number of $(b,c)$ that $b \geq c$ and $b+c = x$, so $c \in [1 \dots \lfloor\frac{x}{2}\rfloor]$.
On the other had if $x > a$ as $b \leq a$ then $b \in [\lceil\frac{x}{2}\rceil \dots a]$ so we have $max\{a - \lceil\frac{x}{2}\rceil - 1, 0\}$ solutions.

Finally the solver will take $M$ and the pregenerated primitive triplets and for each triplet will generate the all multiple triplets for $k = 1\dots$ were at least $x$ or $y$ is smaller than $M$.
Then for each triplet will compute the rutes using $a = x$ and $b + c = y$, and for $a = y$ and $b + c = x$.

\end{document}
